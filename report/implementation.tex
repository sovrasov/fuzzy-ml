\section{Реализация}
Программная реализация алгоритмов построения и верификации модели выполнена на языке
Python 2.7 с использованием библиотеки Numpy. Исходный код можно найти по ссылке \cite{sources}.
Программа позволяет задавать через командную строку следующие параметры:
\begin{itemize}
  \item начальный параметр для генератора псевдослучайных чисел;
  \item количество кластеров в сети Кохонена;
  \item количество частиц в методе роя частиц;
  \item размер тестовой выборки в методе одного выстрела;
  \item количество разбиений в методе скользящего контроля;
  \item режим работы (<<oneshot>>  или <<crossv>>).
\end{itemize}
\par
В режиме <<oneshot>> качество модели оценивается до и после параметрической
оптимизации по методу одного вытрела, а в процессе оптимизации при нахождении
лучшего значения целевой функции оно выводится на экран.
\par
В режиме <<crossv>> запускается метод скользящего контроля. Построение моделей при
этом осуществляется в разных процессах и может быть выполнено параллельно, если
есть соответствующие аппаратные ресурсы. На экран выводится финальная оценка качества и
оценки качества всех построенных в процессе скользящего контроля моделей.
